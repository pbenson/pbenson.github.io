\documentclass{report}

\usepackage{amsmath} % provides numberwithin (and lots more)
\usepackage{graphicx}
\usepackage[backend=bibtex]{biblatex}
\bibliography{qfrmTechnicalQuestionsDev}


\newtheorem{problem}{}
\numberwithin{problem}{chapter} % important bit
\let\oldroblem\problem
\renewcommand{\problem}{\oldroblem\normalfont}
\newcommand{\ds}{\displaystyle}

\begin{document}

\begin{titlepage}
\begin{center}
 {\huge\bfseries Math 573\\ Lecture Notes\\}
 % ----------------------------------------------------------------
 \vspace{1.5cm}
% {\bfseries Instructor: Sergey Nadtochiy}\\[5pt]
% pbenson@umich.edu\\[14pt]
  % ----------------------------------------------------------------
 \vspace{10cm}
 % ----------------------------------------------------------------
\includegraphics{QFRM_rgb}\\[5pt]
{Department of Mathematics}\\[5pt]
{530 Church Street}\\[5pt]
{Ann Arbor, MI 48109-1043,
 USA}\\
 \vfill

\end{center}
\end{titlepage}

%----------------------
% review
%----------------------
\chapter{Introduction}

The course is divided broadly into three sections:
\begin{enumerate}  
\item Arbitrage pricing and hedging 
\item Optimal investment
\item Risk measurement
\end{enumerate}
Examples of each follow.

\section{Arbitrage Pricing and Hedging}
Let $C(K) \equiv$ the price of a European call option struck at $K$. Suppose that there are call options with strike prices of 1, 2, and 3 as depicted. 

\end{document}
